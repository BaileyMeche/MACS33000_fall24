\documentclass[12pt]{article}
\usepackage{amsmath, graphicx, caption}
\usepackage{amsthm}
\usepackage{amsfonts, xcolor, physics}
\usepackage{amssymb}
\usepackage{mathrsfs}
\usepackage[T1]{fontenc} % for \symbol{92} 
\usepackage{comment}


\addtolength{\oddsidemargin}{-1in}
\addtolength{\evensidemargin}{-1in}
\addtolength{\textwidth}{1.75in}
\addtolength{\topmargin}{-1in}
\addtolength{\textheight}{1.75in}
\newcommand{\contra}{$\rightarrow\leftarrow$}
\newcommand{\tb}{  \textbackslash  }
\newcommand{\bj}{\ \Longleftrightarrow \ }

%bailey meche
\begin{document}
%bailey meche
	\begin{center}
		Assignment 2: Sequences, Limits, Derivatives, and Critical Points\\
        MACSS 33000 1 \\
		Due Tuesday, August 27 \\
        %Bailey Meche
	\end{center}
 
\section{Simplify Logarithms}
Express each of the following as a single logarithm 
\begin{enumerate}
    \item $\log(x) + \log(y) - \log(z)$
    \item $2\log(x) + 1$
    \item $\log(x)-2$
\end{enumerate}
Solution:
\begin{enumerate}
    \item $\log(x) + \log(y) - \log(z)  = \log\left(\frac{xy}{z}\right)$
    \item $2\log(x) + 1 = \log(x^2) + 1 = \log(x^2) + \log(e) = \log(ex^2)$
    \item $\log(x)-2 = \log(x) - 2\log(e) = \log(x) - \log(e^2) = \log\left( \frac{x}{e^2}\right)$
\end{enumerate}

\section{Sequences}
Write down the first three terms of each of the following sequences. In each case, state whether the sequence is an arithmetic progression, a geometric progression, or neither. 
\begin{enumerate}
    \item $u_n = 5 + 3n$
    \item $u_n = 3^n$
    \item $u_n = n3^n$
\end{enumerate}
Solution

\begin{enumerate}
    \item Arithmetic sequence: $u_n = 5 + 3n \to \{n \in \mathbb{N}_{\leq 3}: 8,11,14,...\}$
    \item Geometric sequence: $u_n = 3^n \to \{n \in \mathbb{N}_{\leq 3}: 3,9,27,...\}$
    \item Neither: $u_n = n3^n \to \{n \in \mathbb{N}_{\leq 3}: 3,18, 81,...\}$
\end{enumerate}

\section{Find the limit}
In each of the following cases, state whether the sequence $\{u_n\}$ tends to a limit, and find the limit if it exists:
\begin{enumerate}
    \item $u_n = 1+ \frac{1}{2}n$
    \item $u_n = \left(\frac{1}{2}\right)^n$
    \item $\lim\limits_{x \to -4} \frac{x^2 + 5x + 4}{x^2 + 3x -4}$
\end{enumerate}
Solution

\begin{enumerate}
    \item $u_n = 1+ \frac{1}{2}n$
    \begin{align*}
        \lim\limits_{n \to \infty} u_n &=  \lim\limits_{n \to \infty}(1 + \frac{1}{2}n) 
        \\ &=  \lim\limits_{n \to \infty} 1 +  \frac{1}{2}\lim\limits_{n \to \infty} n \to \infty 
        \\ & \text{Limit does not exist.}
    \end{align*}
    \item $u_n = \left(\frac{1}{2}\right)^n$
    \begin{align*}
         \lim\limits_{n \to \infty} u_n &=  \lim\limits_{n \to \infty} \left(\frac{1}{2}\right)^n 
         \\ &= \lim\limits_{n \to \infty} \frac{1}{2^n} 
         \\ &=0
    \end{align*}
    \item $\lim\limits_{x \to -4} \frac{x^2 + 5x + 4}{x^2 + 3x -4}$
    \begin{align*}
        \lim\limits_{x \to -4} \frac{x^2 + 5x + 4}{x^2 + 3x -4}   &= \lim\limits_{x \to -4} \frac{(x+4)(x+1)}{(x+4)(x-1)}
        \\ &= \lim\limits_{x \to -4} \frac{x+1}{x-1}
        \\ &= \frac{3}{5}
    \end{align*}
\end{enumerate}

\section{Determine convergence or divergence}
Determine whether each of the following sequences converges or diverges. If it converges, find the limit
\begin{enumerate}
    \item $a_n = \frac{3+5n^2}{n+n^2}$
    \item $a_n = \frac{(-1)^{n-1}n}{n^2+1}$
\end{enumerate}
Solution
\begin{enumerate}
    \item $a_n = \frac{3+5n}{n+n^2}$
    \begin{align*}
        \lim\limits_{n \to \infty} a_n &= \lim\limits_{n \to \infty} \frac{3+5n^2}{n+n^2} \\ &= \lim\limits_{n \to \infty} \frac{3}{n(1+n)} + 5\lim\limits_{n \to \infty} \frac{n}{1+n}
        \\ &= 0 + 5(1)
        \\ &= 5
        \\ & \text{This sequence converges to 5.}
    \end{align*}
    \item $a_n = \frac{(-1)^{n-1}n}{n^2+1}$ converges to 0. This is seen by removing the alternating term and observing the dominating terms of the  function $ \frac{n}{n^2+1}$ which converges to 0 regardless of the alternating sign term.
\end{enumerate}

\section{Find more limits}
Given that 
\[ \lim\limits_{x \to a} f(x) =-3, \lim\limits_{x \to a} g(x) =0, \lim\limits_{x\to a} h(x) = 8\]
find the limits that exist. If the limit doesn’t exist, explain why.
\begin{enumerate}
    \item $\lim\limits_{x \to a} [f(x) + h(x)]$
    \item $\lim\limits_{x \to a} \frac{f(x)}{g(x)}$
    \item $\lim\limits_{x \to a} \frac{2f(x)}{h(x) - f(x)}$
\end{enumerate}
Solution

\begin{enumerate}
    \item $\lim\limits_{x \to a} [f(x) + h(x)] = \lim\limits_{x \to a} f(x) + \lim\limits_{x \to a} h(x) = -3 + 8=5$
    \item $\lim\limits_{x \to a} \frac{f(x)}{g(x)}$ does not exist. Limit algebra only allows that $\lim\limits_{x \to a} \frac{f(x)}{g(x)} = \frac{\lim\limits_{x \to a} f(x)}{\lim\limits_{x \to a} g(x)}$ if $g \neq 0 \forall x \in \mathbb{R}$, but since $\lim\limits_{x \to a} g(x) \neq 0$ the limit does not exist.
    \item $\lim\limits_{x \to a} \frac{2f(x)}{h(x) - f(x)} =\frac{2 \lim\limits_{x \to a}  f(x)}{ \lim\limits_{x \to a} h(x) - \lim\limits_{x \to a}  f(x)} = \frac{2(-3)}{(8)- (-3)} = -\frac{6}{11}$
\end{enumerate}

\section{Check for discontinuities}
Which of the following functions are continuous? If not, where are the discontinuities?

\begin{enumerate}
    \item $f(x) = \frac{9x^3 -x}{(x-1)(x+1)}$
    \item $f(x) = e^{-x^2}$
\end{enumerate}
Solution
\begin{enumerate}
    \item $f(x) = \frac{9x^3 -x}{(x-1)(x+1)} $ is discontinuous at its asymptotes $x=1,-1$. 
    \item $f(x) = e^{-x^2}$ is continuous for all $x \in \mathbb{R}$.
\end{enumerate}

\section{Find finite limits }
Find the following finite limits
\begin{enumerate}
    \item $\lim\limits_{x \to 1} \left( \frac{x^4-1}{x-1}\right)$
    \item $\lim\limits_{x \to -4} \left( \frac{x^2 +5x + 4}{x^2 + 3x - 4} \right)$
\end{enumerate}
Solution

\begin{enumerate}
    \item $\lim\limits_{x \to 1} \left( \frac{x^4-1}{x-1}\right) = \lim\limits_{x \to 1} \left( \frac{(x^2+1)(x-1)(x+1)}{x-1}\right)  = \lim\limits_{x \to 1}  (x^2+1)(x+1) =4 $
    \item $\lim\limits_{x \to -4} \left( \frac{x^2 +5x + 4}{x^2 + 3x - 4} \right) = \lim\limits_{x \to -4} \left( \frac{(x+4)(x+1)}{(x+4)(x-1)} \right)=   \lim\limits_{x \to -4} \left( \frac{x+1}{x-1} \right) = \frac{3}{5}$
\end{enumerate}

\section{Find infinite limits}
Find the following infinite limits
\begin{enumerate}
    \item $\lim\limits_{x \to \infty} \left(\frac{9x^2}{x^2+3}\right)$
    \item $\lim\limits_{x \to \infty} \left(\frac{3^x}{x^3}\right)$
\end{enumerate}
Solution
\begin{enumerate}
    \item $\lim\limits_{x \to \infty} \left(\frac{9x^2}{x^2+3}\right)=9\lim\limits_{x \to \infty} \left(\frac{x^2}{x^2+3}\right) = 9(1)=9$
    \item $\lim\limits_{x \to \infty} \left(\frac{3^x}{x^3}\right) \to \infty $
\end{enumerate}

\section{Assessing continuity and differentiable }
For each of the following functions, describe whether it is continuous and/or differentiable at
the point of transition of its two formulas
\begin{enumerate}
    \item $f(x) = \begin{cases}
        x^2 & x \geq 0
        \\ -x^2 & x<0
    \end{cases}
    $
    \item $f(x) = \begin{cases}
        x^3 & x \leq 1
        \\ x & x>1
    \end{cases}$
\end{enumerate}
Solution
\begin{enumerate}
    \item $f(x)$ is continuous and differentiable for all $x\in \mathbb{R}$. 
    \item $f(x)$ is continuous at all points including the critical point. To test for differentiability, we test the continuity of the derivative at the critical point: \[ f'(x) = \begin{cases}
        3x^2 & x \leq 1
        \\ 1 & x>1
    \end{cases}\]
    Since this function $f'(x)$ is not continuous at $x=1$, this function is not differentiable for all $x\in \mathbb{R}.$
\end{enumerate}

\section{Possible derivative}
A friend shows you this graph of a function $f(x)$. Which of the following could be a graph of $f'(x)$? For each graph, explain why or why not it
might be the derivative of $f(x)$.

Solution:
\begin{enumerate}
    \item This could not be a possible graph of $f'(x)$. This is because this graph is constant and negative while the graph of $f(x)$ is positive and increasing at some rate, indicating that its derivative must be at least positive. 
    \item This could not be a possible graph of $f'(x)$ since it is positive and constant. The function $f(x)$ is increasing at an increasing rate, so $f'(x)$ must be positive and increasing.
    \item This could be a possible graph of $f'(x)$ since it is positive and increasing. Since $f(x)$ is increasing at an increasing rate,  this is a likely graph of $f'(x)$.
    \item This could not be a possible graph of $f'(x)$. This is because this graph is negative at some points of $x>0$, implying that $f(x)$ should decrease during this portion. Since it does not, this excludes D from a possible graph of $f'(x)$.
\end{enumerate}
What if the figure below was the graph of $f(x)$? Which of the graphs might potentially be the
derivative of $f(x)$ then?

Solution: The graph of $f'(x)$ for this new $f(x)$ is now graph B. This $f(x)$ is positive and increasing at a constant rate instead of an increasing rate. Graph B is positive and constant, matching our expectations. 

\section{Calculate derivatives}
Differentiate the following functions
\begin{enumerate}
    \item $f(x) = 4x^3 + 2x^2 + 5x + 11$
    \item $y=\sqrt{30}$
    \item $h(t) = \log(9t+1)$
    \item $f(x) = \log(x^2e^x)$
    \item $h(y) = \left(\frac{1}{y^2} - \frac{3}{y^4}\right)(y+5y^3)$
    \item $h(x) = \frac{x}{\log(x)}$
\end{enumerate}
Solution
\begin{enumerate}
    \item $f'(x) = 12x^2 + 4x + 5$
    \item $y=0$
    \item $h(t) = \frac{9}{9t+1}$
    \item $f(x) = \frac{(x^2e^x)'}{x^2e^x} = \frac{x + 2}{x}$
    \item $h(y) = \left(\frac{1}{y^2} - \frac{3}{y^4}\right)(1+15y^2)  +\left(-\frac{2}{y^3} + \frac{12}{y^5}\right) (y+5y^3)$
    \item $h(x) = \frac{1}{\log(x)} - \frac{1}{\log^2(x)}$
\end{enumerate}

\section{Use the product and quotient rules}
Differentiate the following function twice – once using the product rule, and once using the
quotient rule
\begin{align*}
    f(x) &= \frac{x^2 -2x}{x^4 + 6}
\end{align*}
Solution
\begin{align*}
    f'(x) &= \frac{(x^4 + 6)(2x -2) - (x^2-2x)(4x^3)}{(x^4 + 6)^2}
    \\ f'(x) &= (x^2 -2x)\left((x^4 + 6)^{-1}\right)' + (x^2 -2x)'(x^4 + 6)^{-1}
    \\ &= (x^2 -2x)(-1)(x^4 + 6)^{-2}(4x^3) + (2x-2)(x^4 + 6)^{-1}
    \\&= \frac{(x^4 + 6)(2x -2) - (x^2-2x)(4x^3)}{(x^4 + 6)^2}
\end{align*}

\section{Composite functions}
For each of the following pairs of functions $g(x)$ and $h(z)$ write out the composite function
$g(h(z))$ and $h(g(x))$. In each case, describe the domain of the composite function
\begin{enumerate}
    \item $g(x) = x^2 +4, h(z) = 5z-1$
    \item $g(x) = x^3, h(z) = (z-1)(z+1)$ 
\end{enumerate}
Solution
\begin{enumerate}
    \item$g(x) = x^2 +4, h(z) = 5z-1$
    \begin{enumerate}
        \item $g(h(z)) = (5z-1)^2 +4$, $\mathbb{D}: z\in \mathbb{R}$
        \item $h(g(x)) = 5(x^2+4)-1, \mathbb{D}:x\in \mathbb{R}$
    \end{enumerate}
    \item $g(x) = x^3, h(z) = (z-1)(z+1)$ 
    \begin{enumerate}
        \item $g(h(z)) = (z-1)^3(z+1)^3$, $ \mathbb{D}:z\in \mathbb{R} \textbackslash \{-1,1\}$
        \item $h(g(x)) = (x^3-1)(x^3+1), \mathbb{D}: x\in \mathbb{R} \textbackslash \{-1,1\}$
    \end{enumerate}
\end{enumerate}

\section{Chain rule}
Use the chain rule to compute the derivative of the composite functions in the previous section
from the derivatives of the two component functions. Then, compute each derivative directly
using your expression for the composite function. Simplify and compare your answers
\begin{enumerate}
    \item $g(x) = x^2 +4, h(z) = 5z-1$
    \item $g(x) = x^3, h(z) = (z-1)(z+1)$ 
    \item $g(x) = 4x+2, h(z) = \frac{1}{4}(z-2)$
\end{enumerate}
Solution
\begin{enumerate}
    \item $g(x) = x^2 +4, h(z) = 5z-1$
     \begin{enumerate}
         \item Chain rule
         \begin{enumerate}
            \item $g(h(z))' = g'(h(z))h'(z) = 2(5z-1)(5)$
            \item $h(g(x))' = h'(g(x))g'(x) = 5(2x)$
        \end{enumerate}
         \item Direct
         \begin{enumerate}
            \item $g(h(z))' = \frac{d}{dz}( (5z-1)^2 +4) = 2(5z-1)(5)$
            \item $h(g(x))' = \frac{d}{dx} 5(x^2+4)-1 = 5(2x)$
        \end{enumerate}
     \end{enumerate}
     
    \item $g(x) = x^3, h(z) = (z-1)(z+1)$ 
     \begin{enumerate}
         \item Chain rule
         \begin{enumerate}
            \item $g(h(z))' = g'(h(z))h'(z) = 3(z-1)^2(z+1)^2(2z)$
            \item $h(g(x))' = h'(g(x))g'(x) = 2(x^3)(3x^2)$
        \end{enumerate}
         \item Direct
         \begin{enumerate}
            \item $g(h(z))' = (z-1)^3(z+1)^3 =3(z-1)^2(z+1)^3 +(z-1)^3(3)(z+1)^2$
            \item $h(g(x))' = (x^3-1)(x^3+1) = (3x^2)(x^3+1)+(x^3-1)(3x^2)$
        \end{enumerate}
     \end{enumerate}
     
    \item $g(x) = 4x+2, h(z) = \frac{1}{4}(z-2)$
    \begin{enumerate}
         \item Chain rule
         \begin{enumerate}
            \item $g(h(z))' = g'(h(z))h'(z) = 1$
            \item $h(g(x))' = h'(g(x))g'(x) = 1$
        \end{enumerate}
         \item Direct
         \begin{enumerate}
            \item $g(h(z))' = \frac{d}{dz}(z) =1$
            \item $h(g(x))' = \frac{d}{dx}(x)=1$
        \end{enumerate}
     \end{enumerate}
\end{enumerate}
\end{document}
