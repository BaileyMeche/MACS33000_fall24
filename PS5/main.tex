\documentclass[12pt]{article}
\usepackage{amsmath, graphicx, caption}
\usepackage{amsthm}
\usepackage{amsfonts, xcolor, physics}
\usepackage{amssymb}
\usepackage{mathrsfs}
\usepackage[T1]{fontenc} % for \symbol{92} 
\usepackage{comment}


\addtolength{\oddsidemargin}{-1in}
\addtolength{\evensidemargin}{-1in}
\addtolength{\textwidth}{1.75in}
\addtolength{\topmargin}{-1in}
\addtolength{\textheight}{1.75in}
\newcommand{\contra}{$\rightarrow\leftarrow$}
\newcommand{\tb}{  \textbackslash  }
\newcommand{\bj}{\ \Longleftrightarrow \ }

\begin{document}
	\begin{center}
		Assignment 5-6: Functions of several variables and optimization with several variables\\
        MACSS 33000 1 \\
		Due Tuesday, September 4 \\
       Bailey Meche
	\end{center}

Time: 2 hrs

Difficulty: 1

\section{Find first partial derivatives}
Find all of the first partial derivatives of each function.

\begin{enumerate}
    \item $f(x,y) = 3x -2y^4 $
        \begin{align*}
            \frac{\partial}{\partial x}f(x,y) &=3
            \\ \frac{\partial }{\partial y}f(x,y) &= -8y^3
        \end{align*}
    \item $f(x,y) = x^5 + 3x^3y^2 + 3xy^4$
        \begin{align*}
        \frac{\partial}{\partial x}f(x,y) &=5x^4 + 9x^2y^2+3y^4
        \\ \frac{\partial }{\partial y}f(x,y) &= 6x^3y+12xy^3
        \end{align*}
    \item $g(x,y) = xe^{3y}$
        \begin{align*}
            \frac{\partial}{\partial x}g(x,y) &=e^{3y}
        \\ \frac{\partial }{\partial y}g(x,y) &=3xe^{3y} 
        \end{align*}
    \item $k(x,y) = \frac{x-y}{x+y}$
    \begin{align*}
        \frac{\partial }{\partial x}k(x,y) &= \frac{2y}{(x+y)^2}
        \\ \frac{\partial }{\partial y}k(x,y) &= \frac{-2x }{(x+y)^2}
    \end{align*}
    \item $h(x,y,z)= x^2e^{yz}$
        \begin{align*}
            \frac{\partial }{\partial x}h(x,y,z) &= 2xe^{yz}
            \\ \frac{\partial }{\partial y}h(x,y,z) &= zx^2 e^{yz}
            \\ \frac{\partial }{\partial z}h(x,y,z) &= yx^2 y^{yz}
        \end{align*}
\end{enumerate}

\section{Find the gradient}
Find the gradient $\grad f$ of the following functions and evaluate them at the given points
\begin{enumerate}
    \item $f(x,y) = \sqrt{x^2+y^2}, \quad (x,y)=(3,4)$
    \[ \grad f = \begin{bmatrix}
        x(x^2+y^2)^{-1/2} \\  y(x^2+y^2)^{-1/2}
    \end{bmatrix} \to \grad f (x,y) = \begin{bmatrix}
        \frac{3}{5} \\ \frac{4}{5}
    \end{bmatrix}\]
    \item $f(x,y,z) = (x+z)e^{x-y}, \quad (x,y,z)=(1,1,1)$
    \[ \grad f = \begin{bmatrix}
        (x+z+1)e^{x-y} \\ -(x+z)e^{x-y} \\ e^{x-y}
    \end{bmatrix} \to \grad f(1,1,1) = \begin{bmatrix}
        3 \\ -2 \\ 1    \end{bmatrix}\]
\end{enumerate}

\section{Find the Hessian}
Find the Hessian $\mathbb{H}$ for the following functions 

\begin{enumerate}
    \item $g(x,y) = x^4 -3x^2y^3$
    \[ \begin{cases}
        \frac{\partial g}{\partial x} = 4x^3 - 6xy^3
        \\ \frac{\partial g}{\partial y} = -9x^2y^2 
    \end{cases} \to \mathbb{H} g = \begin{bmatrix}
        \frac{\partial^2 g}{\partial x^2} & \frac{\partial^2 g}{\partial x\partial y} \\ \frac{\partial^2 g}{\partial x\partial y}& \frac{\partial^2 g}{\partial y^2}
    \end{bmatrix} = \begin{bmatrix}
        12x^2 -6y^3 & -18xy^2 \\ -18xy^2 & -18x^2y
    \end{bmatrix}
    \]
    \item $f(x,y,z) = xyz-x^2$
    \[ \begin{cases}
        \frac{\partial f }{\partial x} = yz - 2x
        \\ \frac{\partial f}{\partial y}=xz
        \\ \frac{\partial f}{\partial z} = xy
    \end{cases} \to \mathbb{H} f = \begin{bmatrix}
        -2 & z&y \\ z&0&x \\ y&x&0
    \end{bmatrix}\]
\end{enumerate}

\section{Find the critical points }
Find the local minimum values, local maximum values, and saddle point(s) of the function. Remember the
process we discussed in class: Calculate the gradient, set it equal to zero to solve the system of equations,
calculate the Hessian, and assess the Hessian at critical values. Be sure to show your work on each of these
steps.4
\begin{enumerate}
    \item $f(x,y) = x^4 + y^4 -4xy +2$
    \begin{align*}
        \grad f(x,y) &= \begin{bmatrix}
            4x^3 -4y \\ 4y^3 -4x        \end{bmatrix} = \Vec{0} \quad \to \quad \begin{cases}
                x^3 =y
                \\ y^3 =x  
            \end{cases} \quad ; \quad \mathbb{H}f(x,y) = \begin{bmatrix}
                12x^2 & -4 \\ -4 & 12y^2
            \end{bmatrix}
    \end{align*}
\begin{center}
\begin{tabular}{ c |c |c |c }
 $(x_0,y_0)$ & $f_{xx} f(x_0,y_0)$ & $\det \mathbb{H}f(x_0,y_0)$ & Interpretation
 \\  \hline \hline
 $(0,0)$ &  0 & -16 & Saddle point\\  
 $(1,1) $& 12 & 128 & Local minimum \\  
 $(-1,-1)$&  12 & 128 & Local minimum 
 \end{tabular}
\end{center}

    \item $k(x,y)=(1+xy)(x+y)$
    \[ \grad k = \begin{bmatrix}
        (1+xy) + y(x+y) \\(1+xy) + x(x+y) 
    \end{bmatrix} = \Vec{0} \quad \to \quad \begin{cases}
        y^2 + 2xy +1=0
        \\ x^2 + 2xy +1=0
    \end{cases} \quad ; \quad \mathbb{H}f(x,y) = \begin{bmatrix}
                2y & 2x+2y \\ 2x+2y & 2x  \end{bmatrix}\]

\begin{center}
\begin{tabular}{ c |c |c |c }
 $(x_0,y_0)$ & $f_{xx} k(x_0,y_0)$ & $\det \mathbb{H}k(x_0,y_0)$ & Interpretation
 \\  \hline \hline
 $(-1,1)$ &  2 &-4 & Saddle point\\  
 $(1,-1) $&  -2 & -4 & Saddle point
 \end{tabular}
\end{center}
\end{enumerate}

\section{Definite integrals}
Solve the following definite integrals using the antiderivative method.5
For all these problems, the basic approach to compute the definite integral of $f(x)$ from a to b is by using the
formula $F(b) - F(a)$, where $F(x)$ is the antiderivative of $f$.
\begin{enumerate}
    \item $\int_6^8 x^3 dx = \frac{1}{4}x^4 \bigg|_6^8  = \frac{1}{4}(8^4-6^4)=700$
    \item $\int_{-1}^0 (3x^2 -1) dx = x^3 -x \bigg|_{-1}^0= 0-(-1-(-1))=0$
    \item $\int_0^1 x^{\frac{3}{7}}dx = \frac{7}{10} x^{\frac{10}{7}} \bigg|_0^1 = \frac{7}{10}$
    \item $\int_1^2 t^{-2}dt = -t^{-1} \bigg|_1^2 = -(\frac{1}{2}-1) = \frac{1}{2}$
    \item $\int_2^4 e^y dy = e^y \bigg|_2^4 = e^4-e^2$
    \item $\int_8^9 2^x dx = \int_8^9 e^{x\log(2)} dx= \frac{1}{\log(2)}\int_{8\log(2)}^{9\log(2)}e^udu = \frac{2^x}{\log(2)}\bigg|_8^9 = \frac{2^8}{\log(2)}$
    \item $\int_3^3 \sqrt{x^5+2} dx=0$ by the Fundamental Theorem of Calculus
\end{enumerate}

\section{Applied integration}
Who traveled the farthest? The least far? 
\begin{align*}
    d_A(t) &= \int_{t_0}^{t_1} v_A(t) dt = \int_0^2 (2t^4+t)dt = \frac{2}{5}t^5 + \frac{1}{2}t^2 \bigg|_0^2 = \frac{74}{5}
    \\ d_B(t) &= \int_{t_0}^{t_1} v_B(t) dt = \int_0^4 4\sqrt{t}dt = 4\cdot \frac{2}{3} \cdot t^{\frac{3}{2}}\bigg|_0^4 = \frac{64}{3}
    \\  d_C(t) &= \int_{t_0}^{t_1} v_C(t) dt = \int_0^{20}2e^{-t}dt = -2e^{-t}\bigg|_0^{20} =  -2e^{-20}+2
\end{align*}
Student B travelled the farthest and Student C travelled the least far. 

\section{Indefinite integrals}
Calculate the following indefinite integrals
\begin{enumerate}
    \item $\int (x^2 - x^{-\frac{1}{2}})dx = \frac{1}{3}x^3 -2x^{\frac{1}{2}}+C$
    \item $\int 360t^6dt = \frac{360}{7}t^7+C $
    \item $\int 2x \log(x^2) dx = 4\int x\log(x)dx$
    \begin{align*}
        \begin{cases}
            u=\log(x) & du = \frac{1}{x}dx
            \\ dv = x dx & v =\frac{1}{2}x^2+C
        \end{cases}
        \\ \int 2x \log(x^2) dx &= 4\left( \frac{1}{2}x^2 \log(x) - \frac{1}{2} \int x dx\right)
        \\ &=  4\left( \frac{1}{2}x^2 \log(x) - \frac{1}{4} x^2\right)
        \\ &= x^2(\log(x^2)-1)
    \end{align*}
\end{enumerate}

\section{Determining convergence}
Determine whether each integral is convergent or divergent. Evaluate those that are convergent.
\begin{enumerate}
    \item $\int_1^\infty \frac{1}{9x^2}dx = \lim\limits_{b \to \infty} \int_1^b \frac{1}{9x^2}dx = -\frac{1}{9}\lim\limits_{b \to \infty} (b^{-1}-1) = \frac{1}{9} \to $ converges
    \item $\int_0^\infty \cos(x)dx = \lim\limits_{b \to \infty}\int_0^b \cos(x)dx =\lim\limits_{b \to \infty} \sin(x) \to $ does not converge
    \item $\int_0^\infty e^{-x}dx = \lim\limits_{b \to \infty}\int_0^b e^{-x}dx=   \lim\limits_{b \to \infty} (-e^{-b}+1)=1 \to $ converges
    \item $\int_{-\infty}^0 x^3 dx = \lim\limits_{b \to -\infty}\int_{b}^0 x^3 dx = \lim\limits_{b \to -\infty} -\frac{1}{4}b^4 \to$ does not converge
\end{enumerate}

\section{More integrals}
Calculate the following integrals
\begin{enumerate}
    \item $\int_0^1 \int_2^3 x^2 y^3 dxdy = \int_0^1 \left(\frac{19}{3}\right)y^3 dy = \frac{19}{12}$
    \item $\int_2^3 \int_0^1 x^2y^3 dydx = \frac{1}{4} \int_2^3 x^2 = \frac{1}{4} \cdot \frac{1}{3}(3^3-2^3)= \frac{19}{12}$
    \item $\int_0^1 \int_0^{\sqrt{1-x^2}}2x^3y dydx = \int_0^1 x^3(1-x^2)dx= \frac{1}{4}x^4- \frac{1}{6}x^6\bigg|_0^1 = \frac{1}{12}$
\end{enumerate}
\end{document}
