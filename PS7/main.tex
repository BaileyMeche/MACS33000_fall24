\documentclass[12pt]{article}
\usepackage{amsmath, graphicx, caption}
\usepackage{amsthm}
\usepackage{amsfonts, xcolor, physics}
\usepackage{amssymb}
\usepackage{mathrsfs}
\usepackage[T1]{fontenc} % for \symbol{92} 
\usepackage{comment}


\addtolength{\oddsidemargin}{-1in}
\addtolength{\evensidemargin}{-1in}
\addtolength{\textwidth}{1.75in}
\addtolength{\topmargin}{-1in}
\addtolength{\textheight}{1.75in}
\newcommand{\contra}{$\rightarrow\leftarrow$}
\newcommand{\tb}{  \textbackslash  }
\newcommand{\bj}{\ \Longleftrightarrow \ }

\begin{document}
	\begin{center}
		Assignment 7: Sample space and probability\\
        MACSS 33000 1 \\
		Due Tuesday, September 5 \\
       Bailey Meche
	\end{center}

\section{Sets}
Consider rolling a six-sided die. Let A be the set of outcomes where the roll is an even number. Let B be the
set of outcomes where the roll is greater than 3. Calculate the sets on both sides of De Morgan’s Laws


\textbf{Solution}

Let 
\begin{enumerate}
    \item $A = \{2,4,6\}$
    \item $B = \{4,5,6\}$
\end{enumerate}
Then, we have 
\begin{align*}
    A \cup B &= \{2,4,5,6\} \\
    (A \cup B)^c &= \{1,3\}
    \\ &= \{1,3\} = \{1,3,5\} \cap \{1,2,3\}
    \\ &= A^c \cap B^c
\end{align*}

\section{Ghostbusting}
Twenty ghostbusters are on their annual camping retreat. Two of them, Abe and Betty, have discovered
that another pair, Candace and Dan, are in fact ghosts posing as ghostbusters. Abe and Betty hatch a plan:
When all 20 campers are sitting in a circle around the campfire, Abe will fire his proton pack at Candace,
and Betty will simultaneously fire her proton pack at Dan, annihilating the ghosts. However, if two proton
streams cross (that is, if the paths of their weapons intersect), it means the end of all life on Earth.
If the ghostbusters are arranged randomly around the fire, what are the chances that Abe and Betty will
cross streams?

\textbf{Solution}

By fixing $A$, we calculate the probability that the stream from Betty to Dan intersects that of Abe to Candace. We calculate this by taking the possible arrangements of members around the circle:
\[BCD,BDC,CBD,CDB,DBC,DCB\]
of which the outcomes $BCD,DCB$ result in intersections. For each of these arrangements, the point $C$ may be arbitrarily chosen. This results in symmetric arrangement possibilities around the 20 seats for $B$ and $D$. Therefore, we may simplify the possible arrangements down to the relative arrangements given above. This gives the probability that the streams will intersect
\[ \Pr(BCD \text{ or } DCB) = \frac{\{BCD,DCB\}}{\{BCD,BDC,CBD,CDB,DBC,DCB\}} = \frac{1}{3}.\]

 \section{Calculate probabilities in a sample space }

 Events A and B are contained within a sample space $S$. Given that $\Pr(A) = 0.5$, $\Pr(B) = 0.3$ and $\Pr(A \cap B) = 0.1$, find
 \begin{enumerate}
     \item $\Pr(A \cup B) = \Pr(A) + \Pr(B) - \Pr(A\cap B) = 0.5+0.3-0.1=0.7$
     \item $\Pr(A \cap B^c) =\Pr(A) - \Pr(A \cap B) = 0.5 - 0.1 = 0.4$
     \item $\Pr\left[(A \cap B^c) \cup (B \cap A^c)\right] =\left(\Pr(A \cap B^c) - \pr(A \cap B)\right) + \left(\Pr(A \cap B^c)-\Pr(A \cap B)\right) = 0.5 - 0.1 + 0.3 - 0.1 = 0.6 $
 \end{enumerate}

\section{Silly campaigns, polling is for political scientists}
A political campaign in New Haven, CT decides to conduct an “experiment” to determine the effectiveness of
knocking on a door in turning a resident of that house out to vote. The campaign foolishly denies an offer
from a team of political scientists to help them design a protocol for this experiment, and instead directs
their two teams of volunteers to each select a random group of the 120 total houses in the district and to go
knock on as many of those random doors as they can in the week before the election. The campaign manager
directs the teams to count the number of doors on which they knock and to record the names of the residents
who live in each house, but neglects to ensure that the two teams select a mutually exclusive set of houses, or
to set bounds on how many houses each team chooses.
On election day, the Team 1 members return, and proudly report to the campaign manager that they knocked
on 70\% of the doors in the electoral district. The Team 2 members return shortly after, and report that they
knocked on 40\% of the doors in the electoral district. In looking over the names the teams recorded, the
campaign manager quickly determines that not only was every house in the district contacted, but some
houses were contacted by both teams. (This will make drawing inferences about the effectiveness of door
knocking . . . difficult.)
Use what we have learned about probability to determine how many houses had their doors knocked on by
both teams.

\textbf{Solution}

Team 1 knocked on 70\% of 120 = 84 houses.
Team 2 knocked on 40\% of 120 = 48 houses.
The number of houses polled by both teams is $(81+48)-120 = 12 = 10\%$ houses. 

\section{Rolling the dice}
We roll two fair 6-sided dice. Each one of the 36 possible outcomes is assumed to be equally likely.
\begin{enumerate}
    \item Find the probability that doubles are rolled.
    \[ \Pr\left\{(1,1),(2,2),...,(6,6)\right\} = \frac{6}{36} = \frac{1}{6}\]
    \item Given that the roll results in a sum of 4 or less, find the conditional probability that doubles are rolled
    \begin{align*}
        \text{Let $A,B$ be the two results of the die}&
        \\ \Pr(A+B\leq 4) &= \Pr\left\{(1,1),(1,2),(1,3),(2,1),(2,2),(3,1) \right\} = \frac{1}{6}
        \\ \Pr(\text{doubles}) &= \frac{1}{6}
        \\ \Pr(\text{doubles} \bigg| A+B\leq 4) &= \frac{\Pr(\text{doubles} \cap A+B\leq 4)}{\Pr(A+B\leq 4)} 
        = \frac{\Pr\left\{(1,1),(2,2)\right\}}{\frac{1}{6}} 
        = \frac{2}{36} \cdot \frac{6}{1} = \frac{1}{3}
    \end{align*}
    \item Find the probability that at least one die roll is a 6.
    \[ \Pr\left\{(6,1),(6,2),...,(6,6),(1,6),...,(5,6)\right\} = \frac{6+55}{36}=\frac{11}{36}\]
    \item Given that the two dice land on different numbers, find the conditional probability that at least one die
roll is a 6.
    \begin{align*}
        \Pr(A \neq B) &= \Pr(\text{doubles}^c) = 1-\frac{1}{6}=\frac{5}{6}
        \\ \Pr(\text{one 6} \bigg| A \neq B) &= \frac{\Pr(\text{one 6}) \textbackslash \{(6,6)\}}{\frac{5}{6}} = \frac{1}{3}
    \end{align*}
\end{enumerate}

\section{A two-envelope puzzle}
The release of two out of three prisoners has been announced, but their identity is kept secret. One of the
prisoners considers asking a friendly guard to tell him who is the prisoner other than himself that will be
released, but hesitates based on the following rationale: at the prisoner’s present state of knowledge, the
probability of being released is 2/3, but after he knows the answer, the probability of being released will
become 1/2, since there will be two prisoners (including himself) whose fate is unknown and exactly one of.
the two will be released. What is wrong with this line of reasoning

\textbf{Solution}

Regardless of prior knowledge, the prisoner's probability of release remains $\frac{2}{3}$. The decision by which the prisoners are released were not conditional on one of the other 2 prisoners being in the released group. As far as we know, each of the prisoners have an equal probability of being released. 

\section{Survey Says}
A survey has 52\% respondents 50 or older and 48\% respondents under 50. Within the survey, on a particular
question, 9.5\% of the 50-plus population agrees strongly while 1.7\% of under 50 respondents agree strongly.

\begin{enumerate}
    \item What is the probability someone selected at random is 50 or older?
    \[ 52\%\]
    \item  The selected individual strongly agrees with the survey question. Now what is the likelihood that
person is 50 or older? Explain your reasoning and SHOW ALL YOUR WORK
    \begin{align*}
        \Pr(\text{Age}\geq 50 \bigg| \text{Agrees}) &= \frac{\Pr(\text{Age} \geq 50 \cap \text{Agree}) }{\Pr(\text{Agree}) }
        \\ &= \frac{9.5\%}{9.5\% + 1.7\%} \approx 81.2\%
    \end{align*}
    \item  Are the two answers above the same or different? Explain.

    They are different since conditional knowledge was provided. 
    \item (for fun, no points) What is the survey question?

    "Is the US economy getting better?"
\end{enumerate}

\end{document}
